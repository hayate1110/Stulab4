\documentclass[a4paper, 11pt, titlepage]{jsarticle} 
\usepackage[dvipdfmx]{graphicx}
\usepackage{amsmath}
\usepackage{listings}
\usepackage{comment}
\usepackage{url}
\usepackage{here}

\title{2023-知能情報基礎演習\\4-2 無線通信における降雨減衰特性のモデリング} 
\date{提出日: \today\\提出期限日: 2023年12月19日\\実験日: 2023年11月5日, 12日}
\author{提出者氏名: 城間 颯\\提出者学籍番号: 225719A\\担当教員名: 宮里 智樹}

\begin{document}
\maketitle
\tableofcontents 
\clearpage 

\section{1週目}
\subsection{課題1}
mattermost からデータをダウンロードし、中身を確認した。

RainDataのフォルダ構成を以下に示す。

\begin{verbatim}
RainData
├── 20090601
│   └── 20090601_rain.csv
├── 20090602
│   └── 20090602_rain.csv
├── 20090603
│   └── 20090603_rain.csv
├── 20090604
│   └── 20090604_rain.csv
...
├── 20091230
│   └── 20091230_rain.csv
└── 20091231
		└── 20091231_rain.csv
\end{verbatim}

RxDataのフォルダ構成を以下に示す。

\clearpage
\begin{verbatim}
RxData
├── 200906
│   ├── 20090601
│   │  ├── 192.168.100.11_csv.log
│   │   └── 192.168.100.9_csv.log
│   ├── 20090602
│   │   ├── 192.168.100.11_csv.log
│   │   └── 192.168.100.9_csv.log
│   ├── 20090603
│   │   ├── 192.168.100.11_csv.log
│   │   └── 192.168.100.9_csv.log
│   ...
│
├── 200910
│   ├── 20091001
│   │   ├── 192.168.100.11_csv.log
│   │   └── 192.168.100.9_csv.log
│   ...
│
├── 200911
│   ├── 20091101
│   │   ├── 192.168.100.11_csv.log
│   │   └── 192.168.100.9_csv.log
│   ...
│
└── 200912
    ├── 20091201
    │   ├── 192.168.100.11_csv.log
    │   └── 192.168.100.9_csv.log
    ...
\end{verbatim}

\subsection{課題2}
2秒毎に記録されているRainDataの測定値と、1分毎に記録されているRxDataの測定値を、10秒毎のデータに作りかえた。

前処理済みのデータを、元データとフォルダ構成は同じにして、RxData\_fixと、RainData\_fixという名前のフォルダに保存した。

RainDataの前処理を行うスクリプトを、listings\ref{pg:pp_rain}に示す。

RxDataの前処理を行うスクリプトを、listings\ref{pg:pp_rx}に示す。

スクリプトは、Pythonを用いて作成した。

データの欠損値は、直前のデータで置き換えた。直前のデータがない場合は、直後のデータで置き換えた。

\lstinputlisting[language=Python, numbers=left, breaklines=true, 
basicstyle=\ttfamily\footnotesize, frame=single, caption=RainDataの前処理用スクリプト, 
label=pg:pp_rain]{pp_RainData.py}

\lstinputlisting[language=Python, numbers=left, breaklines=true, 
basicstyle=\ttfamily\footnotesize, frame=single, caption=RxDataの前処理用スクリプト, 
label=pg:pp_rx]{pp_RxData.py}

\section{2週目}
\subsection{手順1}
1分間降雨強度を1時間(60分)降雨強度に変換した。

変換後のデータは、元データとフォルダ構成は同じにして、RainData\_unit\_convという名前のフォルダに保存した。

計算式は以下を用いた。雨粒の量をcとする。
$$
S \lbrack mm/h \rbrack = c \times 0.0083333 \times 60
$$

雨粒量から、1時間(60分)降雨強度に変換するスクリプトをlistings\ref{pg:unit_rain}に示す。

\lstinputlisting[language=Python, numbers=left, breaklines=true, 
basicstyle=\ttfamily\footnotesize, frame=single, caption=RainDataの単位変換スクリプト, 
label=pg:unit_rain]{unit_conv_RainData.py}

\subsection{手順2}
RxDataの、18GHzの場合の、受信強度の生データを、物理量$\lbrack dB \rbrack$に変換した。

変換後のデータは、元データとフォルダ構成は同じにして、RxData\_unit\_convという名前のフォルダに保存した。

計算式は以下を用いた。受信電界元の値をEとする。
\begin{align*}
	P [dB] = 
	\begin{cases}
		E \div 2 - 121 & \text{if $E \geq 0$}\\
		(E + 256) \div 2 - 121 & \text{if $E < 0$}\\
	\end{cases}
\end{align*}

RxDataを単位変換するスクリプトをlistings\ref{pg:unit_rx}に示す。

\lstinputlisting[language=Python, numbers=left, breaklines=true, 
basicstyle=\ttfamily\footnotesize, frame=single, caption=RxDataの単位変換スクリプト, 
label=pg:unit_rx]{unit_conv_RxData.py}

\subsection{手順3}
頻度分布を求めるにあたって、基準となる数値を決めた。

1時間降雨強度の最大値は、143.499426[mm/h]。最小値は、0.0[mm/h]であった。

したがって、RainDataの、1時間降雨強度の出現頻度を求める最大値は、144[mm/h]。最小値は、0[mm/h]とした。
また、刻み間隔は3[mm/h]とした。

18GHzの場合の受信強度の最大値は、-173.0[dB]。最小値は、-229.5[dB]であった。

したがって、RxDataの、18GHzの場合の、受信強度の出現頻度を求める最大値は、0[dB]。最小値は、-230[dB]とした。
また、刻み間隔は-3[dB]とした。

26GHzの場合の受信強度の最大値は、-1.0[dB]。最小値は、-99.0[dB]であった。

したがって、RxDataの、26GHzの場合の、受信強度の出現頻度を求める最大値は、0[dB]。最小値は、-100[dB]とした。
また、刻み間隔は-3[dB]とした。

\subsection{手順4}
前処理と単位変換済みのRainDataから、3[mm/h]の刻み間隔ごとに、当てはまるデータが何個あったのかを数え、頻度分布を作成した。

同様に、18GHzの場合と、26GHzの場合の、RxDataでも、-3[dB]の刻み間隔ごとに、当てはまるデータが何個あったのかを数え、頻度分布を作成した。

\subsection{手順5}
頻度分布を足し合わせて累積分布を作成した。

RainDataの頻度分布を求めるプログラムを、listings\ref{pg:dist_rain}に示す。

RxDataの、18GHzの場合の、頻度分布を求めるプログラムを、listings\ref{pg:dist_rx_18}に示す。

RxDataの、26GHzの場合の、頻度分布を求めるプログラムを、listings\ref{pg:dist_rx_26}に示す。

頻度分布を、pandasのDataFrameとして、変数に保存していたため、pandasのcumsum()を用いて、累積和を計算した。

また、RainDataの$0[mm/h] \sim 3[mm/h]$および、RxDataの$0[dB] \sim -3[dB]$の区間のデータは、4ヶ月の総時間数、1051000[10秒]に置き換えた。

\lstinputlisting[language=Python, numbers=left, breaklines=true, 
basicstyle=\ttfamily\footnotesize, frame=single, caption=RainDataのグラフ作成スクリプト, 
label=pg:dist_rain]{make_dist_RainData.py}

\lstinputlisting[language=Python, numbers=left, breaklines=true, 
basicstyle=\ttfamily\footnotesize, frame=single, caption=RxData(18GHz)のグラフ作成スクリプト, 
label=pg:dist_rx_18]{make_dist_RxData_18.py}

\lstinputlisting[language=Python, numbers=left, breaklines=true, 
basicstyle=\ttfamily\footnotesize, frame=single, caption=RxData(26GHz)のグラフ作成スクリプト, 
label=pg:dist_rx_26]{make_dist_RxData_26.py}

\subsubsection{手順6}
matplotlibを用いて、累積分布のグラフを作成した。

累積分布を求めるプログラムは、頻度分布を求めるプログラムと一緒に記述した。

\subsection{手順7}
横軸を時間から、パーセンテージに直した。
各データの累積時間(単位: 10秒)を、4ヶ月の総時間数、1051000[10秒]で割り、結果に100をかけて、パーセンテージを出した。

計算式を以下に示す。各データの累積時間(単位: 10秒)をtとする。
$$
r [\%] = \frac{t}{1051000} \times 100
$$

\subsection{手順8}
グラフの横軸を対数軸とし、片対数グラフとした。

降雨強度累積時間分布を図\ref{fig:rain_dist}に示す。

18GHzの場合の、受信強度累積時間分布を図\ref{fig:rx_dist_18}に示す。

26GHzの場合の、受信強度累積時間分布を図\ref{fig:rx_dist_26}に示す。

\begin{figure}[H]
	\centering
	\includegraphics[]{rain_dist.png}
	\caption{降雨強度累積時間分布}
	\label{fig:rain_dist}
\end{figure}

\begin{figure}[H]
	\centering
	\includegraphics[]{rx_dist_18.png}
	\caption{受信強度累積時間分布(18GHz)}
	\label{fig:rx_dist_18}
\end{figure}

\begin{figure}[H]
	\centering
	\includegraphics[]{rx_dist_26.png}
	\caption{受信強度累積時間分布(26GHz)}
	\label{fig:rx_dist_26}
\end{figure}

\section*{参考文献・引用文献}

\begin{thebibliography}{99}
\bibitem{stulab4} 宮里 智樹 . “2023-知能情報基礎演習 4-2 無線通信における降雨減衰特性のモデリング . $2023 知能情報基礎演習4 授業資料$ .
\end{thebibliography}

\end{document}