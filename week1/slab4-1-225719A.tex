\documentclass[a4paper, 11pt, titlepage]{jsarticle} 
\usepackage[dvipdfmx]{graphicx}
\usepackage{amsmath}
\usepackage{listings}
\usepackage{comment}
\usepackage{url}
\usepackage{here}


\title{2023-知能情報基礎演習\\4-1 微分方程式の数値解法} 
\date{提出日: \today\\提出期限日: 2023年12月5日\\実験日: 2023年11月28日}
\author{提出者氏名: 城間 颯\\提出者学籍番号: 225719A\\担当教員名: 宮里 智樹}

\begin{document}
\maketitle
\tableofcontents 
\clearpage 

\section{問題1}
常微分方程式に初期値を与えることで、積分定数などの未知数を含まない形で、元の関数を求めることができる。
このような問題を初期値問題と呼ぶ。
オイラー法は、初期値問題を数値的に解く方法である。
例として、以下のような常微分方程式の初期値問題を考える。
$$
\frac{dy}{dx} = f(x,y), y(x_0) = y_0
$$
この常微分方程式の解$y=y(x)$が図\ref{fig:y}のような曲線になっているとする。

\begin{figure}[H]
	\centering
	\includegraphics[width=100mm]{y.png}
	\caption{y=y(x)のグラフ}
	\label{fig:y}
\end{figure}

微分方程式を数値的に解くとは、初期値$x_0$から始めて、$x_1=x_0+h, x_2=x_1+h, \dots$における$y(x)$の値を順次求めていくことに対応する。
このときのhを刻み幅と呼ぶ。

最も単純に考えると、今yのxに関する微分$\frac{dy}{dx}$、すなわち曲線の傾きがf(x,y)なので、
$$
y(x_1) = y(x_0) + hf(x_0,y_0)
$$
のように近似すれば、十分に小さいhに対しては、それなりに良い解を得られると考えられる。
$y(x_1)$をオイラー法で求めたときの様子を図\ref{fig:y_euler}に示す。

これを一般化して、
$$
y(x_i) = y(x_{i-1}) + hf(x_{i-1},y_{i-1})
$$
という漸化式にしたがって、順次$y(x_1), y(x2), \dots$を求めていく方法を、オイラー法または前進オイラー法と呼ぶ。

オイラー法は刻み幅hを十分に小さくすることで、高い精度を得られるはずだが、実際には、
\begin{itemize}
	\item hを小さくすると、計算回数が増え、計算量が増加する
	\item hを小さくすると、丸め誤差が大きくなる
	\item 計算回数が増えると、誤差の累積も増加する
\end{itemize}
といった問題があり、それほど良い精度は得られない。

\section{問題2}
初期値問題
$$
\frac{dx}{dt} = -x, x(0) = 1
$$
を解く。

\begin{align*}
	\frac{dx}{dt} &= -x\\
	dx &= -xdt\\
	\int \frac{1}{x}dx &= -\int dt\\
	\ln x &= -t + C\\
	x &= e^{-t + c} = e^{c}e^{-t} = C'e^{-t}\\
	x(0) &= C'e^{-0} = C' = 1\\
\end{align*}
よって、$x = e^{-t}$。

\section{実験:野球ボールの軌道計算}
Listing\ref{pg:euler}に、実験で使用したオイラー法による野球ボールの軌道計算のプログラムを示す。

このプログラムは、オイラー法により、野球ボールの軌道計算を行うプログラムで、
\begin{align*}
	v_{i+1} &= v_{i} + \tau \cdot (\frac{1}{m} F_a(v_{i}) - g\hat{y})\\
	r_{i+1} &= r_{i} + \tau \cdot v_{i}\\
	\\
	F_a(v) &= -\frac{1}{2}C_d\rho A|v|v\\
\end{align*}
の漸化式にしたがって、各時刻の、速度と位置の値を計算している。

\lstinputlisting[language=C, numbers=left, breaklines=true, 
basicstyle=\ttfamily\footnotesize, frame=single, caption=オイラー法による野球ボールの軌道計算のプログラム, 
label=pg:euler]{baseball.cpp}

\section{問題3}
$$
\frac{dy}{dx} = d(x,y), y(x_0) = y_0
$$
の初期値問題を考える。

オイラー法では、初期値問題の解y(x)はわからず、現在わかっている関数y(x)上の点$(x_i, y_i)$を接点とするy(x)の接線を求め、
その接線が、点$(x_i, y_i)$の近くでは、関数y(x)を近似的に表しているとみなして、$x_{i+1}$における接線の高さ$y_{i+1}'$を、真の値である$y(x_{i+1})$の近似値であると考える。
この時のイメージを図\ref{fig:y_euler}に示す。

$\tau$は、$x_{i+1}とx_i$の差を表す。
したがって、$\tau$を大きくするほど、$x_{i+1}とx_i$は離れるため、接線から求める点$(x_{i+1}, y_{i+1})$も、接点$(x_i, y_i)$から離れていく。
ゆえに、$\tau$を大きくするほど、接線が、真の関数y(x)を近似的に表すという仮定は成り立ちにくくなるため、真の値と、オイラー法で求めた近似値は、誤差が大きくなることが考えられる。

\begin{figure}[H]
	\centering
	\includegraphics[width=150mm]{y_euler.png}
	\caption{y=y(x)のオイラー法による近似}
	\label{fig:y_euler}
\end{figure}

\section{問題4(実験結果)}
時間刻み$\tau$は、0.2、0.15、0.1、0.05、0.01、0.005、0.001の7つを選んだ。

初期値は、高さ0m、初期速度40m/s、投射角度45度、空気抵抗ありとした。

$\tau = 0.2$の時のグラフを、図\ref{fig:tau_0.2}に示す。

$\tau = 0.15$の時のグラフを、図\ref{fig:tau_0.15}に示す。

$\tau = 0.1$の時のグラフを、図\ref{fig:tau_0.1}に示す。

$\tau = 0.05$の時のグラフを、図\ref{fig:tau_0.05}に示す。

$\tau = 0.01$の時のグラフを、図\ref{fig:tau_0.01}に示す。

$\tau = 0.005$の時のグラフを、図\ref{fig:tau_0.005}に示す。

$\tau = 0.001$の時のグラフを、図\ref{fig:tau_0.001}に示す。

x軸は、ボールの水平方向への移動距離(m)を、y軸は、ボールの軌道の高さ(m)を表す。

理論値というラベルがついた曲線は、空気抵抗なしの場合の、解析的に計算したボールの軌道である。

実測値というラベルがついた曲線は、オイラー法により求めた、ボールの軌道である。

\begin{figure}[H]
	\centering
	\includegraphics[width=100mm]{tau_0.2.png}
	\caption{$\tau = 0.2$の時のボールの軌道}
	\label{fig:tau_0.2}
\end{figure}

\begin{figure}[H]
	\centering
	\includegraphics[width=100mm]{tau_0.15.png}
	\caption{$\tau = 0.15$の時のボールの軌道}
	\label{fig:tau_0.15}
\end{figure}

\begin{figure}[H]
	\centering
	\includegraphics[width=100mm]{tau_0.1.png}
	\caption{$\tau = 0.1$の時のボールの軌道}
	\label{fig:tau_0.1}
\end{figure}

\begin{figure}[H]
	\centering
	\includegraphics[width=100mm]{tau_0.05.png}
	\caption{$\tau = 0.05$の時のボールの軌道}
	\label{fig:tau_0.05}
\end{figure}

\begin{figure}[H]
	\centering
	\includegraphics[width=100mm]{tau_0.01.png}
	\caption{$\tau = 0.01$の時のボールの軌道}
	\label{fig:tau_0.01}
\end{figure}

\begin{figure}[H]
	\centering
	\includegraphics[width=100mm]{tau_0.005.png}
	\caption{$\tau = 0.005$の時のボールの軌道}
	\label{fig:tau_0.005}
\end{figure}

\begin{figure}[H]
	\centering
	\includegraphics[width=100mm]{tau_0.001.png}
	\caption{$\tau = 0.001$の時のボールの軌道}
	\label{fig:tau_0.001}
\end{figure}

\section{問題5}
\subsection{ホイン法}
$$
\frac{dy}{dx} = d(x,y), y(x_0) = y_0
$$
の初期値問題を考える。
y(x+h)の値は、関数yの点xの周りのテイラー展開で、
\begin{align*}
	y(x+h) &= y(x) + \frac{y'(x)}{1!}((x+h) - x) + \frac{y''(x)}{2!}((x+h) - x)^2 + \dots + \frac{y^{(n)}(x)}{n!}((x+h) - x)^n + \dots\\
	&= y(x) + \frac{h}{1!}y'(x) + \frac{h^2}{2!}y''(x) + \dots + \frac{h^n}{n!}y^{(n)}(x) + \dots
\end{align*}
と表せる。
これを1階微分の項まで用いた時、
$$
y(x+h) = y(x) + \frac{h}{1!}y'(x) = y(x) + hf(x,y)
$$
となる。

すなわち、オイラー法は微分方程式の解をhの一次式で近似しているといえる。

したがって、精度を上げるには、2次以上の項を用いて、テイラー級数近似を行えば良い。

テイラー展開を二次項まで利用した場合を考える。

y(x+h)の近似は、
$$
y(x+h) \approx y(x) + \frac{h}{1!}y'(x) + \frac{h^2}{2!}y''(x)
$$
と表せる。

ここで、f(x,y)のxによる微分を、適当な刻み幅khで差分法により計算すると、
\begin{align*}
	\frac{df(x,y)}{dx} &\approx \frac{f(x+kh, y(x+kh)) - f(x,y)}{kh}\\
	&\approx \frac{f(x+kh, y(x) + khf(x,y)) - f(x,y)}{kh}
\end{align*}
と書けるので、これを元の式に代入すると、
$$
y(x+h) = y(x) + (1 - \frac{1}{2k})hf(x,y) + \frac{h}{2k}f(x+kh, y(x) + khf(x,y))
$$
という公式が得られる。

この公式において、k=1と置いた時に得られる
$$
y(x+h) = y(x) + \frac{h}{2}f(x,y) + \frac{h}{2}f(x+kh, y(x) + hf(x,y))
$$
という公式をホイン法または、改良オイラー法と呼ぶ。

\subsection{ホイン法の実装}
今回の実験の、初期値問題は、
\begin{align*}
	\frac{dv}{dt} &= f(v) = \frac{1}{m}F_a(v) - g\hat{y}, F_a(v) = -\frac{1}{2}C_d\rho A|v|v\\
	\frac{dv}{dt} &= v\\
\end{align*}
である。

ホイン法で、$v(t + \tau)$の値を求める式は、
\begin{align*}
	v(t + \tau) &= v(t) + \frac{\tau}{2}(k_1 + k_2)\\
	k_1 &= \frac{dv}{dt} = f(v)\\
	k_2 &= \frac{d^2}{dt^2}v = f(v(t) + \tau f(v)) = f(v + tk_1)\\
\end{align*}
である。

$r(t + \tau)$を求める式は、
\begin{align*}
	r(t + \tau) = r(t) + \tau v(t)\\
\end{align*}
である。

ホイン法を実装したプログラムを、Listing\ref{pg:heun}に示す。

このプログラムは、上で求めた、ホイン法による漸化式にしたがって、各時刻の、速度と位置の値を計算している。

\lstinputlisting[language=C, numbers=left, breaklines=true, 
basicstyle=\ttfamily\footnotesize, frame=single, caption=ホイン法による野球ボールの軌道計算のプログラム, 
label=pg:heun]{baseball_heun.cpp}

\subsection{ホイン法の結果}
時間刻み$\tau$は、0.2、0.15、0.1、0.05、0.01、0.005、0.001の7つを選んだ。

初期値は、高さ0m、初期速度40m/s、投射角度45度、空気抵抗ありとした。

$\tau = 0.2$の時のグラフを、図\ref{fig:tau_0.2_heun}に示す。

$\tau = 0.15$の時のグラフを、図\ref{fig:tau_0.15_heun}に示す。

$\tau = 0.1$の時のグラフを、図\ref{fig:tau_0.1_heun}に示す。

$\tau = 0.05$の時のグラフを、図\ref{fig:tau_0.05_heun}に示す。

$\tau = 0.01$の時のグラフを、図\ref{fig:tau_0.01_heun}に示す。

$\tau = 0.005$の時のグラフを、図\ref{fig:tau_0.005_heun}に示す。

$\tau = 0.001$の時のグラフを、図\ref{fig:tau_0.001_heun}に示す。

x軸は、ボールの水平方向への移動距離(m)を、y軸は、ボールの軌道の高さ(m)を表す。

理論値というラベルがついた曲線は、空気抵抗なしの場合の、解析的に計算したボールの軌道である。

実測値というラベルがついた曲線は、オイラー法により求めた、ボールの軌道である。

\begin{figure}[H]
	\centering
	\includegraphics[width=100mm]{tau_0.2_heun.png}
	\caption{ホイン法で求めた$\tau = 0.2$の時のボールの軌道}
	\label{fig:tau_0.2_heun}
\end{figure}

\begin{figure}[H]
	\centering
	\includegraphics[width=100mm]{tau_0.15_heun.png}
	\caption{ホイン法で求めた$\tau = 0.15$の時のボールの軌道}
	\label{fig:tau_0.15_heun}
\end{figure}

\begin{figure}[H]
	\centering
	\includegraphics[width=100mm]{tau_0.1_heun.png}
	\caption{ホイン法で求めた$\tau = 0.1$の時のボールの軌道}
	\label{fig:tau_0.1_heun}
\end{figure}

\begin{figure}[H]
	\centering
	\includegraphics[width=100mm]{tau_0.05_heun.png}
	\caption{ホイン法で求めた$\tau = 0.05$の時のボールの軌道}
	\label{fig:tau_0.05_heun}
\end{figure}

\begin{figure}[H]
	\centering
	\includegraphics[width=100mm]{tau_0.01_heun.png}
	\caption{ホイン法で求めた$\tau = 0.01$の時のボールの軌道}
	\label{fig:tau_0.01_heun}
\end{figure}

\begin{figure}[H]
	\centering
	\includegraphics[width=100mm]{tau_0.005_heun.png}
	\caption{ホイン法で求めた$\tau = 0.005$の時のボールの軌道}
	\label{fig:tau_0.005_heun}
\end{figure}

\begin{figure}[H]
	\centering
	\includegraphics[width=100mm]{tau_0.001_heun.png}
	\caption{ホイン法で求めた$\tau = 0.001$の時のボールの軌道}
	\label{fig:tau_0.001_heun}
\end{figure}

\section{考察}
グラフに理論値として示した曲線は、空気抵抗なしの場合のボールの軌道である。一方、実測値として示した曲線は、空気抵抗ありの場合のボールの軌道である。
二つの曲線を比較すると、随分飛距離に差があることがわかる。
2つの大きな違いは、空気抵抗があるか、ないかであるから、この結果に影響を及ぼしたのは、空気抵抗であろうと考えられる。

果たして、空気抵抗の影響とは、こんなに大きいものだろうか。空気抵抗の影響の大きさを、計算式から考えてみる。

ここで、オイラー法による、速度の計算式は、
\begin{align*}
	v_{i+1} &= v_{i} + \tau \cdot (\frac{1}{m} F_a(v_{i}) - g\hat{y})\\
	F_a(v) &= -\frac{1}{2}C_d\rho A|v|v\\
\end{align*}
である。

この式の大まかな意味を考えると、$F_a$は負の値であるから、$v_{i+1}$は、$v_i$から、$F_a$の値に比例した大きさの値を引いた値、であることがわかる。

ここで、$F_a$の式は、変数の$|v|v$に係数をかけた式と見ることができるが、$|v|$は、プログラム中、L2ノルムで定義されているから、直角三角形の斜辺が常に、3辺のうち、長さが最大であることを考えると、$|v| > v$が成り立つ。したがって、$|v|v > v^2$が成り立つ。

よって、単純化して考えると、$F_a$は、vの2乗に比例した値よりも大きい値を出力する関数である。
すなわち、空気抵抗は、vの2乗に比例した値よりも大きい値であると考えることができる。

ゆえに、このような大きな値が空気抵抗として、ボールの速度を減衰させるから、空気抵抗がない場合と、空気抵抗がある場合の飛距離に、大きな差が出たのではないだろうか。

\section*{参考文献・引用文献}

\begin{thebibliography}{99}
\bibitem{stulab4} 宮里 智樹 . “2023-知能情報基礎演習 4-1 微分方程式の数値解法” . $2023 知能情報基礎演習4 授業資料$ .
\bibitem{tokyo} 大竹 豊, AN Qi . “微分方程式の数値解法” . $東京大学工学部 精密工学科 プログラミング応用 I・ II 授業資料$ . \url(http://www.den.t.u-tokyo.ac.jp/ad_prog/ode/) , (\today ) .
\end{thebibliography}

\end{document}